\documentclass[12pt,-letter paper]{article}
\usepackage{siunitx}
\usepackage{setspace}
\usepackage{gensymb}
\usepackage{xcolor}
\usepackage{caption}
%\usepackage{subcaption}
\doublespacing
\singlespacing
\usepackage[none]{hyphenat}
\usepackage{amssymb}
\usepackage{relsize}
\usepackage[cmex10]{amsmath}
\usepackage{mathtools}
\usepackage{amsmath}
\usepackage{commath}
\usepackage{amsthm}
\interdisplaylinepenalty=2500
%\savesymbol{iint}
\usepackage{txfonts}
%\restoresymbol{TXF}{iint}
\usepackage{wasysym}
\usepackage{amsthm}
\usepackage{mathrsfs}
\usepackage{txfonts}
\let\vec\mathbf{}
\usepackage{stfloats}
\usepackage{float}
\usepackage{cite}
\usepackage{cases}
\usepackage{subfig}
%\usepackage{xtab}
\usepackage{longtable}
\usepackage{multirow}
%\usepackage{algorithm}
\usepackage{amssymb}
%\usepackage{algpseudocode}
\usepackage{enumitem}
\usepackage{mathtools}
%\usepackage{eenrc}
%\usepackage[framemethod=tikz]{mdframed}
\usepackage{listings}
%\usepackage{listings}
\usepackage[latin1]{inputenc}
%%\usepackage{color}{   
%%\usepackage{lscape}
\usepackage{textcomp}
\usepackage{titling}
\usepackage{hyperref}
%\usepackage{fulbigskip}   
\usepackage{tikz}
\usepackage{graphicx}
\lstset{
  frame=single,
  breaklines=true
}
\let\vec\mathbf{}
\usepackage{enumitem}
\usepackage{graphicx}
\usepackage{siunitx}
\let\vec\mathbf{}
\usepackage{enumitem}
\usepackage{graphicx}
\usepackage{enumitem}
\usepackage{tfrupee}
\usepackage{amsmath}
\usepackage{amssymb}
\usepackage{mwe} % for blindtext and example-image-a in example
\usepackage{wrapfig}
\graphicspath{{figs/}}
\providecommand{\mydet}[1]{\ensuremath{\begin{vmatrix}#1\end{vmatrix}}}
\providecommand{\myvec}[1]{\ensuremath{\begin{bmatrix}#1\end{bmatrix}}}
\providecommand{\cbrak}[1]{\ensuremath{\left\{#1\right\}}}
\providecommand{\brak}[1]{\ensuremath{\left(#1\right)}}
\begin{document}
\begin{enumerate}
	\item  We are given a positive interger $r$ and a rectangular board $ABCD$ with dimensions $\mydet{AB} =20$, $\mydet{BC}=12$. The rectangle is divided into a grid of \(20\times12\) unit squares. The following moves are permitted on the board: one can move from one square to another only if the distance between the centers of the two squares is $\sqrt{r}$. The task is to find a sequence of moves leading from the square with $A$ as a vertex to the square with $B$ as a vertex.
\begin{enumerate}
\item Show that the task cannot be done if $r$ is divisible by $2$ or $3$.
\item Prove that the task is possible when $r=73$.
\item Can the task be done when $r=97$?
\end{enumerate}
\item Let $P$ be a point inside triangle $ABC$ such that
	\begin{align*}
\angle{APB}-\angle{ACB}=\angle{APC}-\angle{ABC}.
	\end{align*}
Let $D$, $E$ be the incenters of triangles $APB$ ,$APC$, respectively. Show that $AP$ ,$BD$, $CE$ meet at a point.
\item Let $S$ denote the set of nonnegative integers. Find all functions $f$ from $s$ to itself such that 
	\begin{align*}
		f(m+f(n))=f(f(m))+f(n)         
		\forall{m}, n \epsilon S. 
	\end{align*}
\item The positive integers $a$ and $b$ are such that the numbers $15a+16b$ and $16a-15b$ are both squares of positive integers. What is the least possible value that can be taken on by the smaller of these two squares?
\item Let $ABCDEF$ be a convex hexagon such that $AB$ is parallel to $DE$, $BC$ is parallel to $EF$, and $CD$ is parallel to $FA$. Let $R_A$, $R_C$, $R_E$ denote the circumradii of triangles $FAB$, $BCD$, $DEF$, respectively, and let $P$ denote the perimeter of the hexagon. Prove that
	\begin{align*}
		R_A+R_C+R_E\geq\frac{p}{2}.
	\end{align*}
\item Let $p$, $q$, $n$ be three positive integers with $p+q<n$. Let $(x_0, x_1,....,x_n)$ be an $(n+1)$-tuple of integers satisfying the following conditons:
	\begin{enumerate}
\item $x_0=x_n=0$.
\item For each $i$ with $1\leq{i}\leq{n}$, either $x_i-x_{i-1}=p$ or $x_i-x_{i-1}=-q$.\\ 
Show that there exist indices $i < j$ with $(i,j) \neq(0,n)$, such that $x_i=x_j$.
	\end{enumerate}
\end{enumerate}
    \end{document}
