\documentclass[12pt,-letter paper]{article}
\usepackage{siunitx}
\usepackage{setspace}
\usepackage{gensymb}
\usepackage{xcolor}
\usepackage{caption}
%\usepackage{subcaption}
\doublespacing
\singlespacing
\usepackage[none]{hyphenat}
\usepackage{amssymb}
\usepackage{relsize}
\usepackage[cmex10]{amsmath}
\usepackage{mathtools}
\usepackage{amsmath}
\usepackage{commath}
\usepackage{amsthm}
\interdisplaylinepenalty=2500
%\savesymbol{iint}
\usepackage{txfonts}
%\restoresymbol{TXF}{iint}
\usepackage{wasysym}
\usepackage{amsthm}
\usepackage{mathrsfs}
\usepackage{txfonts}
\let\vec\mathbf{}
\usepackage{stfloats}
\usepackage{float}
\usepackage{cite}
\usepackage{cases}
\usepackage{subfig}
%\usepackage{xtab}
\usepackage{longtable}
\usepackage{multirow}
%\usepackage{algorithm}
\usepackage{amssymb}
%\usepackage{algpseudocode}
\usepackage{enumitem}
\usepackage{mathtools}
%\usepackage{eenrc}
%\usepackage[framemethod=tikz]{mdframed}
\usepackage{listings}
%\usepackage{listings}
\usepackage[latin1]{inputenc}
%%\usepackage{color}{   
%%\usepackage{lscape}
\usepackage{textcomp}
\usepackage{titling}
\usepackage{hyperref}
%\usepackage{fulbigskip}   
\usepackage{tikz}
\usepackage{graphicx}
\lstset{
  frame=single,
  breaklines=true
}
\let\vec\mathbf{}
\usepackage{enumitem}
\usepackage{graphicx}
\usepackage{siunitx}
\let\vec\mathbf{}
\usepackage{enumitem}
\usepackage{graphicx}
\usepackage{enumitem}
\usepackage{tfrupee}
\usepackage{amsmath}
\usepackage{amssymb}
\usepackage{mwe} % for blindtext and example-image-a in example
\usepackage{wrapfig}
\graphicspath{{figs/}}
\providecommand{\mydet}[1]{\ensuremath{\begin{vmatrix}#1\end{vmatrix}}}
\providecommand{\myvec}[1]{\ensuremath{\begin{bmatrix}#1\end{bmatrix}}}
\providecommand{\cbrak}[1]{\ensuremath{\left\{#1\right\}}}
\providecommand{\brak}[1]{\ensuremath{\left(#1\right)}}
\begin{document}
\begin{enumerate}
\item Let $A$,$B$,$C$,$D$ be four distinct points on a line, in that order. The circles with diameters $AC$ and $BD$ intersect at $X$ and $Y$. The line $XY$ meets $BC$ at $Z$. Let $P$ be a point on the line $XY$ other than $Z$. The line $CP$ intersects the circle with diameter $AC$ at $C$ and $M$, and the line $BP$ intersects the circle with diameter $BD$ at $B$ and $N$. Prove that the lines $AM$, $DN$, $XY$ are concurrent.
\item Let $a$, $b$, $c$ be positive real numbers such that $abc=1$. Prove that \\ 
\begin{align*}	
\frac{1}{a^3(b+c)}+\frac{1}{b^3(c+a)}+\frac{1}{c^3(a+b)}\geq\frac{3}{2}.
\end{align*}
\item Determine all integers $n>3$ for which there exist $n$ points $A_1, . . . .,A_n$ in the plane, no three collinear, and real numbers $r_1,...,r_n$ such that for $1\leq{i}<{j}<{k}\leq{n}$, the area of $\triangle A_iA_jA_k$ is $r_i+r_j+r_k$.
\item Find the maximum value of $x_{0}$ for which there exists a sequence $x_{0},x_{1}.....x_{1995}$ of positive reals with $x_{0}=x_{1995}$, such that for $i=1,....,1995$,
\begin{align*}
	x_{i-1}+\frac{2}{x_{i-1}}=2x_i+\frac{1}{x_i}. 
\end{align*}
\item Let $ABCDEF$ be a convex hexagon with $AB=BC=CD$ and $DE=EF=FA$, such that $\angle{BCD}=\angle{EFA}={\pi}/{3}$. Suppose $G$ and $H$ are points in the interior of the hexagon such that $\angle{AGB}=\angle{DHE}={2\pi}/{3}$. Prove that $AG+GB+GH+DH+HE\geq CF$.
\item Let $p$ be an odd prime number. How many p-element subsets $A$ of $\{1,2,....2p\}$ are there, the sum of whose elements is divisible by $p$? 
\end{enumerate}
\end{document}
