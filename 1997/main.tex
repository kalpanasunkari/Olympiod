\documentclass[12pt,-letter paper]{article}
\usepackage{siunitx}
\usepackage{setspace}
\usepackage{gensymb}
\usepackage{xcolor}
\usepackage{caption}
%\usepackage{subcaption}
\doublespacing
\singlespacing
\usepackage[none]{hyphenat}
\usepackage{amssymb}
\usepackage{relsize}
\usepackage[cmex10]{amsmath}
\usepackage{mathtools}
\usepackage{amsmath}
\usepackage{commath}
\usepackage{amsthm}
\interdisplaylinepenalty=2500
%\savesymbol{iint}
\usepackage{txfonts}
%\restoresymbol{TXF}{iint}
\usepackage{wasysym}
\usepackage{amsthm}
\usepackage{mathrsfs}
\usepackage{txfonts}
\let\vec\mathbf{}
\usepackage{stfloats}
\usepackage{float}
\usepackage{cite}
\usepackage{cases}
\usepackage{subfig}
%\usepackage{xtab}
\usepackage{longtable}
\usepackage{multirow}
%\usepackage{algorithm}
\usepackage{amssymb}
%\usepackage{algpseudocode}
\usepackage{enumitem}
\usepackage{mathtools}
%\usepackage{eenrc}
%\usepackage[framemethod=tikz]{mdframed}
\usepackage{listings}
%\usepackage{listings}
\usepackage[latin1]{inputenc}
%%\usepackage{color}{   
%%\usepackage{lscape}
\usepackage{textcomp}
\usepackage{titling}
\usepackage{hyperref}
%\usepackage{fulbigskip}   
\usepackage{tikz}
\usepackage{graphicx}
\lstset{
  frame=single,
  breaklines=true
}
\let\vec\mathbf{}
\usepackage{enumitem}
\usepackage{graphicx}
\usepackage{siunitx}
\let\vec\mathbf{}
\usepackage{enumitem}
\usepackage{graphicx}
\usepackage{enumitem}
\usepackage{tfrupee}
\usepackage{amsmath}
\usepackage{amssymb}
\usepackage{mwe} % for blindtext and example-image-a in example
\usepackage{wrapfig}
\graphicspath{{figs/}}
\providecommand{\mydet}[1]{\ensuremath{\begin{vmatrix}#1\end{vmatrix}}}
\providecommand{\myvec}[1]{\ensuremath{\begin{bmatrix}#1\end{bmatrix}}}
\providecommand{\cbrak}[1]{\ensuremath{\left\{#1\right\}}}
\providecommand{\brak}[1]{\ensuremath{\left(#1\right)}}
\begin{document}
\begin{enumerate}
\item In the plane the points with integer coordinates are the vertices of unit squares. The squares are colored alternately black and white (as on a chessboard).\\                                                                                           For any pair of positive integers $m$ and $n$, consider a right-angled triangle whose vertices have integer coordinates and whose legs, of lengths $m$ and $n$, lie along edges of the squares.\\                                                             Let $S_1$ be the total area of the black part of triangle ans $S_2$ be the total area of white part. Let
\begin{align*}
f(m,n)=\mydet{S_1-S_2}.                                                             
\end{align*}                                                                        \begin{enumerate}                                                                   \item calculate $f(m,n)$ for all positive integers $m$ and $n$ which are either both even or both odd.
\item Prove that $f(m,n) \leq \frac{1}{2}max\cbrak{m,n}$ for all $m$ and $n$.
\item Show that there is no constant $C$ such that $f(m,n)<c$ for all $m$ and $n$.
\end{enumerate}
\item The angle at $A$ is the smallest angle of triangle $ABC$. The point $B$ and $C$ divide the circumcircle of the triangle into two arcs. Let $U$ be an interior point of the arc between $B$ and $C$ which does not contain $A$. The perpendicular bisectors of $AB$ and $AC$ meet the line $AU$ at $V$ and $W$, respctively. The lines $BV$ and $CW$ meet at $T$. Show that 
\begin{align*}
AU=TB+TC.
\end{align*}
\item Let $x_1,x_2,....,x_n$ be the real numbers satisfying the conditions 
\begin{align*}
\mydet{x_1+x_2+....+x_n}=1
\end{align*}
and
\begin{align*}
\mydet{x_i}\leq\frac{n+1}{2}      i=1,2,...,n.
\end{align*}
Show that there exists a permutation $y_1, y_2,....,y_n$ of $x_1, x_2,...,x_n$ such that
\begin{align*}
\mydet{y_1+2y_2+.....+ny_n}\leq\frac{n+1}{2}.
\end{align*}
\item A \(n\times n\) matrix whose entires come from the set $S = \cbrak{1,2,..,2n-1}$ is called a silver matrix if, for each $i=1,2,...n,$ the ith row and ith column together contain all elements of $S$. Show that 
\begin{enumerate}
\item there is no silver matrix for $n=1997$;
\item silver matrices exist for infinitely many values of $n$.
\end{enumerate}
\item Find all pairs $(a,b)$ of integers $a,b \geq 1$ that satisfy the equation 
\begin{align*}
a^{b^2}=b^{a}.
\end{align*}
\item For each positive integer $n$, let $f(n)$ denote the number of ways of representing $n$ as a sum of powers of $2$ with nonnegative integer exponents. Representations which differ only in the ordering of their of their summands are considered to be the same. For instance, $f(4)=4$, because the number $4$ can be represented in the following four ways;
\begin{align*}
4;2 + 2;2 + 1 + 1;1 + 1 + 1 + 1.
\end{align*}
Prove that, for any integer $n \geq{3}$,
\begin{align*}
2^{n^2/4} < f(2^n)<2^{n^2/2}.
\end{align*}
\end{enumerate}
\end{document} 
